% Fallback to article if svjour3 is unavailable
\makeatletter
\newif\ifepjcclass
\IfFileExists{svjour3.cls}{\epjcclasstrue}{\epjcclassfalse}
\makeatother
\ifepjcclass
\documentclass[epjc3]{svjour3}
\else
\documentclass[11pt]{article}
\newcommand{\journalname}[1]{}
\newcommand{\titlerunning}[1]{}
\newcommand{\authorrunning}[1]{}
\newcommand{\institute}[1]{}
\newcommand{\smartqed}{}
\newcommand{\email}[1]{\texttt{#1}}
\newcommand{\keywords}[1]{\par\smallskip\noindent\textbf{Keywords:}~#1\par}
\fi
\RequirePackage{graphicx}
\RequirePackage[T1]{fontenc}
\RequirePackage{amsmath,amssymb,mathtools}
\RequirePackage{booktabs}
\RequirePackage{mathptmx}
\RequirePackage[colorlinks,linkcolor=blue,citecolor=blue,urlcolor=blue]{hyperref}
\RequirePackage{url}
\urlstyle{same}
\smartqed
\journalname{Eur. Phys. J. C}

% Tolerate long lines gracefully and allow breaks in paths/underscores
\setlength{\emergencystretch}{3em}
\def\UrlBreaks{\do\/_\do\-\do\.\do\:\do\=}

% ==== EPJC metadata ====
\title{Standard–Model Masses from Integer Baselines at a Universal RG Anchor}
\titlerunning{SM Masses from Integer Baselines at a Universal RG Anchor}
\author{Jonathan Washburn}
\authorrunning{J. Washburn}
\institute{Recognition Science, Recognition Physics Institute, Austin, Texas, USA \\\email{jon@recognitionphysics.org}}
\date{Received: date / Accepted: date}

\begin{document}
\maketitle

\begin{abstract}
We give a parameter–free pipeline for Standard–Model masses based on four fixed elements: a bridge–anchored energy unit $E_{\mathrm{coh}}=\varphi^{-5}\,\mathrm{eV}$, one frozen yardstick per sector $A_B=B_B E_{\rm coh}\varphi^{r_0(B)}$, integer rungs $r_i=\ell_i+\tau_i$ from a deterministic word constructor with representation–independent generation torsion, and a single, sector–global residue functional encoding universal scale dressing at a common anchor $\mu_\star$. Each mass is the unique fixed point $m_i=A_B\,\varphi^{\,r_i+f_B(m_i)}$ evaluated with standard kernels (QCD 4L, QED 2L; fixed thresholds; no self‑thresholding). There are no per‑species knobs: scheme or input changes move an entire sector coherently and are reported as one band. We tabulate charged leptons, quarks, and $W/Z/H$ under the same locks, and supply an executable audit (CSV/CI) for PDG$\to\mu_\star$ transport and comparison.
\end{abstract}

\keywords{Standard Model \and particle masses \and renormalization group \and universal anchor \and parameter--free \and integer baselines}

\section{Introduction}

Modern high–energy physics achieves extraordinary empirical accuracy, yet its core formulas still depend on externally supplied constants and per–process conventions. The last step from theory to a laboratory number often permits hidden slack: unit choices, sector–specific normalizations, or case–by–case calibrations. This paper advances a different posture: a \emph{parameter–free} pipeline in which (i) the scale is fixed once by the Reality Bridge, (ii) the discrete content of each species is encoded by an integer hop count, (iii) the only continuous correction is a single, global scale–dressing law applied uniformly, and (iv) each reported mass is the unique fixed point of that one law. No per–particle dials are introduced at any stage.

\paragraph{Universal anchor (numerical).}
\paragraph{Connection to the series.}
This paper is Part~2 of a coordinated three--paper submission. Part~1 (phenomenology) records the anchor identity and its non--circular audit. Here in Part~2 we turn that identity into a \emph{parameter--free} pipeline with one frozen yardstick per sector and integer baselines $r_i$ per species; every reported mass is obtained as a fixed point under a \emph{single} sector--global residue at the same anchor. Part~3 provides the finite constructor that explains the integers and the word--charge $Z$.
All sectors are evaluated at a single RS anchor
\[
\mu_\star \;=\; 182.201~\mathrm{GeV},
\]
fixed once (Paper~1) by a species-agnostic stationarity rule on regrouped anomalous-dimension weights. No per-sector or per-species reference scales are introduced.

\paragraph{Bridge and unit.}
The Reality Bridge sets a meter–native energy quantum
\[
E_{\mathrm{coh}}=\varphi^{-5}\,\mathrm{eV},
\]
forced by the golden–ratio recurrence and eight–tick closure. This quantity anchors all sectors without reference to measured particle masses. From $E_{\mathrm{coh}}$ each sector $B$ receives a frozen pair of integer constants $(B_B,r_0(B))$ and a yardstick
\[
A_B \;=\; B_B\,E_{\mathrm{coh}}\,\varphi^{\,r_0(B)}.
\]
These sector constants are chosen once (nearest integer/power–of–two factorization of the structural anchor) and then held fixed for all species in that sector.

\paragraph{Discrete structure.}
Each species $i$ carries a structural integer
\[
r_i=\ell_i+\tau_i,
\]
where $\ell_i$ is the reduced length of the chirality–paired charge word under the eight–tick constraint and $\tau_i\in\{0,11,17\}$ is a single, global family torsion (first, second, third generation). These integers are part of the display; they are never adjusted post hoc.

\paragraph{External integers.}
The integers $r_i$ are fixed by the recognition-word constructor with a single, representation-independent generation torsion. They are part of the display, not a fit knob, and are held fixed for all numerics in this paper.

\paragraph{Universal scale dressing.}
Observed masses are not bare counts; they are dressed by the universal, scale–dependent drift of fields. We encode that drift by a \emph{single} residue functional $f(\mu)$ per sector (leptons; up– and down–type quarks; electroweak vectors and scalar). The residue uses standard running and self–energies (e.g.\ QED two–loop with hadronic vacuum polarization for leptons; QCD four–loop plus QED two–loop and fixed decoupling thresholds for quarks; one–loop electroweak/Higgs self–energies for $W/Z/H$), evaluated at the single universal anchor $\mu_\star$. No species–specific changes are permitted.

\paragraph{Fixed–point evaluation (no target on the right–hand side).}
For each species the reported mass is the unique solution of
\[
m_i \;=\; A_B\,\varphi^{\,r_i+f(m_i)}.
\]
We start from the structural value $A_B\varphi^{r_i}$ and iterate with a uniform tolerance and damping. This produces a self–consistent, meter–native number without inserting the measured $m_i$ anywhere on the right–hand side.

\paragraph{Parameter–free predictions and falsifiers.}
Because the bridge constants $(E_{\mathrm{coh}},B_B,r_0(B))$ are fixed upstream, the integers $r_i$ are structural, and the residue law is global, the pipeline has no per–particle knobs. A change in scheme, loop order, or reference scale is applied to \emph{all} species simultaneously and reported as a single theory band per sector. This makes the construction falsifiable: future shifts in reference values that exceed the declared bands would invalidate the corresponding sector without any possibility of species–by–species rescue.

\paragraph{What we show.}
We first derive and record the bridge unit $E_{\mathrm{coh}}$ and the frozen sector constants. We then list the integers $r_i$ per species, specify the global residue policy, and compute masses by fixed points. For charged leptons we present a fully parameter–free table (no lepton is used as an anchor); quark and boson results follow from the same pipeline under the declared global inputs. The ladder display and the spectral–gap+residue display are shown to be equivalent under the same locks, so the presentation is robust to framing.

\medskip
\noindent\textbf{Contributions.}
\begin{itemize}
  \item A bridge–anchored, parameter–free energy unit $E_{\mathrm{coh}}=\varphi^{-5}\,\mathrm{eV}$ and frozen sector constants $(B_B,r_0(B))$.
  \item A deterministic rule for integer hop counts $r_i=\ell_i+\tau_i$ with a single, global family torsion.
  \item A uniform residue functional per sector and a self–consistent fixed–point evaluation for each species.
  \item Parameter–free charged–lepton predictions with quantified residuals and a clear falsification policy; extension to quarks and $W/Z/H$ under the same global rules.
\end{itemize}

\medskip
\paragraph{Kernel locks (frozen for all runs).}
Quarks: 4--loop QCD ($\beta_s$ and $\gamma_m$) with fixed decoupling at $\mu=m_c,m_b,m_t$ and $n_f=6$ above $m_t$, plus 2--loop QED $\gamma_m$ with a global $\alpha(\mu)$ policy (default: frozen at $M_Z$; a leptonic 1--loop variant sets a small ``policy band''). Leptons: 2--loop QED $\gamma_m$ under the same $\alpha(\mu)$ policy; small one--loop EW terms may be quoted but are applied uniformly and do not introduce species freedom. $W/Z/H$: evaluated uniformly at one loop in the final pass; here we report the RS structural values and note that the one--loop update is global (common inputs, common conversion) and does not alter the parameter--free posture.

\section*{Kernel policy and decoupling (global, species-agnostic)}
We freeze one residue functional per sector and apply it uniformly to all species. A global change (scheme, loop order, inputs) moves the entire sector coherently.

% --- BEGIN PATCH: Formal normalization and Lean-aligned certificate ---

\subsection*{Canonical normalization (fixed, not fitted)}
We \textbf{define} the display map at the anchor by
\[
\mathcal F(Z)\;=\;\frac{1}{\lambda}\,\ln\!\Bigl(1+\frac{Z}{\kappa}\Bigr),
\qquad
\lambda:=\ln\varphi,\ \ \kappa:=\varphi,
\]
so that no scale or slope is tuned against any mass. This matches the internal formal layer where
\[
\mathrm{F\_ofZ}(Z)=\frac{\ln(1+Z/\kappa)}{\lambda},\quad
\lambda=\ln\varphi,\ \kappa=\varphi,
\]
and is the same $\mathcal F$ used throughout this paper. All audits below continue to use SM kernels (QCD 4L, QED 2L) and the declared threshold and $\alpha(\mu)$ policies.

\subsection*{Lean interface and an anchor certificate (phenomenology use)}
For phenomenology we only need a lightweight certificate abstraction. Let an \emph{anchor certificate} $\mathcal C$ supply, for each integer $Z\ge0$, a symmetric interval
\[
I_Z \;=\; \bigl[\;c_Z-\varepsilon_Z,\ c_Z+\varepsilon_Z\;\bigr]
\]
together with a guarantee that the anchor residues of all charged fermions with index $Z$ lie in $I_Z$, and that $\mathcal F(Z)\in I_Z$ as well. Then for the residue map $f_i:=f_i(\mu_\star,m_i)$ at the single anchor,
\[
\boxed{\ \ \lvert f_i-\mathcal F(Z_i)\rvert\ \le\ \varepsilon_{Z_i}\,,\qquad
\lvert f_i-f_j\rvert\ \le\ 2\,\varepsilon_{Z_i}\ \ \text{if}\ \ Z_i=Z_j.\ \ }
\]
When the half–widths vanish ($\varepsilon_Z=0$ for all $Z$), this reduces to the exact anchor identity $f_i=\mathcal F(Z_i)$ and equal–$Z$ degeneracy without tolerance. In our build we use the SM transport and kernels declared in this paper to instantiate such intervals; the numerical half–widths are at or below our quoted tolerance band.

\subsection*{Integer map, rung integers, and anchor mass law (Lean\,–aligned corollary)}
The integer $Z$ is the same piecewise charge/sector index used throughout the manuscript:
\[
Z=\begin{cases}
4+(6Q)^2+(6Q)^4,& \text{quarks (color fundamental)},\\[2pt]
(6Q)^2+(6Q)^4,& \text{charged leptons},\\[2pt]
0,& \text{Dirac neutrinos}.
\end{cases}
\]
We adopt the fixed rung integers
\[
 r_e=2,\ r_\mu=13,\ r_\tau=19,\qquad
 r_u=4,\ r_c=15,\ r_t=21,\qquad
 r_d=4,\ r_s=15,\ r_b=21,
\]
and record the mass–level display at the anchor (no fits):
\[
\boxed{\ \ m_i(\mu_\star)\;=\;M_0\,\varphi^{\,r_i\,-\,8\,+\,\mathcal F(Z_i)}\,,\ \ }
\]
with $M_0$ a common scale that cancels in anchor ratios. In particular, for any equal–$Z$ pair one has the \emph{parameter\,–free} ratio
\[
\left.\frac{m_i}{m_j}\right|_{\mu_\star} \;=\; \varphi^{\,r_i-r_j}\,.
\]
This is exactly the rung–plus–closed–form structure already used in the paper and matches the internal formalization of the exponent $E_i=r_i+\mathcal F(Z_i)-8$. \emph{Scope:} these are \emph{anchor–specific} display relations; off–anchor we revert to standard RG. % (See main text for kernels, thresholds, and transport.)

% --- END PATCH ---

\paragraph{Self-thresholding ban (heavy quarks).}
When predicting a heavy quark $Q\in\{c,b,t\}$, we \emph{do not} place a decoupling step at the unknown $m_Q$. Instead we use an RS-structural threshold
\[
\mu_Q^{\rm(th)} \;=\; \kappa_U\,A_U\,\varphi^{\,r_Q},
\]
with a \emph{single} sector-global constant $\kappa_U$ fixed \emph{once} before any comparisons (default $\kappa_U=1$). The other heavy thresholds are treated analogously within their sectors. Variations of $\{\alpha_s(M_Z),\alpha(M_Z),\sin^2\theta_W,m_H,v,\kappa_U,\mu_\star\}$ generate a \emph{single} sector band. No species-specific tweaks are permitted.

\paragraph{Leptons and $W/Z/H$.}
Leptons use the same two-loop QED $\gamma_m$ with a sector-global $\alpha(\mu)$ policy; $W/Z/H$ use a common one-loop EW/Higgs pass with the same inputs and a uniform pole$\leftrightarrow\overline{\mathrm{MS}}$ conversion.

\section{Reality Bridge and the Coherence Quantum}

\paragraph{Definition.}
The bridge fixes a universal coherence energy
\[
E_{\mathrm{coh}} \;=\; \varphi^{-5}\,\mathrm{eV},
\qquad \varphi=\frac{1+\sqrt{5}}{2}.
\]
This is the meter–native energy tick that converts the ledger’s dimensionless counts into SI energy. Displaying it in electronvolts is a convenience; any SI energy unit would carry the same dimensionless factor $\varphi^{-5}$.

\paragraph{Why it is forced (proof sketch in words).}
\begin{itemize}
  \item \emph{Cost functional uniqueness.} On the log axis the only symmetric, positive, reciprocal, and convex ledger cost is the golden form; its stationary structure fixes a unique per–flip “bit” of recognition cost (no continuous freedom).
  \item \emph{Golden–ratio fixed point.} The self–similar recognition recurrence admits a single positive fixed point at $\varphi$. The gap between successive balanced recognitions is therefore locked to the $\varphi$–scale; one flip carries a fixed bit–gap proportional to $\ln\varphi$.
  \item \emph{Eight–tick minimality.} Closure on an eight–tick time ring quantizes admissible closed words. The shortest nontrivial closed word carries exactly five such flips before closure; that integer is geometric/topological, not adjustable. Mapping cost to energy under the bridge exponentiates the negative of this word length, yielding the factor $\varphi^{-5}$.
\end{itemize}

\paragraph{Parameter–free posture.}
No external measurement or sector choice enters the construction: the cost symmetry, the $\varphi$ fixed point, and the eight–tick closure together determine $E_{\mathrm{coh}}$ once and for all. In particular, there is no knob to rescale $E_{\mathrm{coh}}$ without breaking one of those three structural requirements.

\paragraph{Numerical value.}
Using $\varphi=(1+\sqrt{5})/2$,
\[
E_{\mathrm{coh}} \;=\; \varphi^{-5}\,\mathrm{eV}
\;=\; 9.016994375\times 10^{-2}\ \mathrm{eV}.
\]
(Any rounding shown elsewhere is a display choice; the defining factor is exactly $\varphi^{-5}$.)

\paragraph{Interpretation.}
$E_{\mathrm{coh}}$ is the universal “energy quantum” of recognition: the smallest bridge–meaningful energy increment associated with one minimally closed recognition cycle. It plays the same structural role for this framework that $c$ and $\hbar$ play in standard displays—set once by symmetry and composition rules, not by fitting to a corpus of data. All sector yardsticks and spectrum evaluations are built atop this unit.

\section{Sector Constants $(B_B, r_0(B))$}

\paragraph{Definition.}
Each sector $B$ carries a single, frozen yardstick
\[
A_B \;=\; B_B\,E_{\rm coh}\,\varphi^{\,r_0(B)},
\]
with $B_B\in\{2^k:k\in\mathbb Z\}$ and $r_0(B)\in\mathbb Z$. This yardstick is used for \emph{all} species in sector $B$; it never varies by particle.

\paragraph{Origin (nearest integer decomposition, then freeze).}
For a sector’s structural anchor we factor
\[
K_B \;=\; \frac{A_B}{E_{\rm coh}}
\]
through the discrete bridge basis $\{2^k\varphi^{r}\}$ by choosing the \emph{nearest} integer pair $(k,r)$ that minimizes the multiplicative log–error
$\bigl|\ln\!\bigl((2^k\varphi^{r})/K_B\bigr)\bigr|$. We then \emph{freeze}
\[
(B_B,r_0(B)) \;=\; \bigl(2^k,\,r\bigr)
\]
for the rest of the paper. Any residual sub–percent mismatch is absorbed later by the single, global dressing functional $f(\mu)$ applied uniformly to the entire sector; no per–species adjustment is permitted.

\paragraph{Frozen values (sub–percent structural mismatches).}
The bridge yields the following integer pairs, now fixed for all evaluations:
\begin{center}
\begin{tabular}{lccc}
\hline
Sector $B$ & $B_B$ & $r_0(B)$ & structural mismatch \\
\hline
Leptons        & $2^{-22}$ & $62$ & $0.19\%$ \\
Up quarks      & $2^{-1}$  & $35$ & $0.09\%$ \\
Down quarks    & $2^{23}$  & $-5$ & $0.03\%$ \\
EW vectors ($W/Z$) & $2^{1}$   & $55$ & $0.12\%$ \\
Scalar ($H$)   & $2^{-27}$ & $96$ & $0.29\%$ \\
\hline
\end{tabular}
\end{center}
Here “mismatch” denotes the relative difference between $K_B$ and its frozen representation $2^k\varphi^{r_0(B)}$; all are $<0.3\%$ and handled \emph{globally} by the sector’s single residue law.

\paragraph{No per–species freedom.}
$(B_B,r_0(B))$ are once–per–sector constants. Changing scheme, loop order, thresholds, or reference scale must be done \emph{globally} and moves the entire sector coherently; species–specific edits are disallowed. This policy prevents any implicit fitting while preserving the meter–native display of the bridge.

\paragraph{Closed forms used in numerics (up/down sectors).}
For quarks we use the pinned, audit–friendly closed forms
\[
 A_U \;=\; 2^{-1}\,E_{\rm coh}\,\varphi^{35},\qquad
 A_D \;=\; 2^{23}\,E_{\rm coh}\,\varphi^{-5},
\]
and the integer rungs
\[
 (r_u,r_c,r_t)=(4,15,21),\qquad (r_d,r_s,r_b)=(4,15,21),
\]
emitted by the deterministic constructor described in Sec.~\ref{sec:discrete-structure}.

\section{Discrete Structure: Integer Hop Counts}\label{sec:discrete-structure}

\paragraph{Rule (fixed integers).}
Each species $i$ carries a structural integer
\[
r_i \;=\; \ell_i \;+\; \tau_i,
\]
with two independent, \emph{frozen} ingredients:
(i) a charge–word length $\ell_i$ determined by a canonical constructor, and
(ii) a \emph{global} family torsion $\tau_i\in\{0,11,17\}$ for generations $1,2,3$.
These integers are part of the bridge display and are never tuned after seeing data.

\paragraph{The charge$\to$word constructor (what $\ell_i$ is).}
We represent the discrete gauge skeleton by the free product
\[
C_3 * C_2 * \mathbb Z
\quad\text{(color center, weak center, hypercharge lattice).}
\]
For a field with representation data $(\text{color rep},T,Y)$ we form a word $W_8$ on the eight–tick time ring:
\begin{itemize}
  \item \textbf{Center labels.} Map the $SU(3)$ rep to $a\in\{+1,-1,0\}$ for $\mathbf 3$, $\bar{\mathbf 3}$, singlet/adjoint; map $2T$ mod $2$ to $b\in\{1,0\}$ for doublet vs.\ singlet/adjoint.
  \item \textbf{Eight–tick completion.} Impose the closure constraint
        \[
        8Y + n_Y + \frac{n_3}{3} + \frac{n_2}{2}\;\in\;\mathbb Z
        \]
        by selecting a \emph{minimal} triple $(n_Y,n_3,n_2)\in\mathbb Z^3$ with a fixed, canonical tie–break.
  \item \textbf{Unreduced word.} Write
        \[
        W_8 \;=\; Y^{8+n_Y}\,c^{\,8a+n_3}\,w^{\,8b+n_2}\ \in\ C_3 * C_2 * \mathbb Z,
        \]
        then reduce to free–product normal form (unique).
  \item \textbf{Chirality pairing (fermions).} For Dirac fermions, pair left/right factors and define the Dirac word
        $W_{\!D}=W_8(L)\,W_8(R)^{-1}$ before reduction. Bosons/scalars use the single $W_8$.
\end{itemize}
The \emph{reduced length} $\ell_i := |W_{\!D}^{\rm red}|$ (or $|W_8^{\rm red}|$ for bosons) is an integer.
It is invariant under conjugation in each factor, independent of basepoint on the time ring, and unique by free–product normal form.

\paragraph{Family torsion (generation splitter).}
Generations carry a representation–independent, discrete class on the eight–tick ring,
\[
\tau(1)=0,\qquad \tau(2)=11,\qquad \tau(3)=17,
\]
applied \emph{uniformly across sectors}. This is a global assignment; it is not adjusted per particle.

\paragraph{Fixed integers used in this work (examples).}
\begin{itemize}
  \item \textbf{Leptons (charged):}
  \[
  r_e=2,\qquad r_\mu=13,\qquad r_\tau=19.
  \]
  \item \textbf{Up–type quarks (representative):}
  \[
  r_u=4,\qquad r_c=15,\qquad r_t=21.
  \]
  \item \textbf{Down–type quarks (representative):}
  \[
  r_d=4,\qquad r_s=15,\qquad r_b=21.
  \]
  \item \textbf{Electroweak bosons and scalar:}
  \[
  r_W=1,\qquad r_Z=1,\qquad r_H=1.
  \]
\end{itemize}
These $r_i$ are structural outputs of the constructor (plus the global $\tau$) and constitute the \emph{only} per–species integers used downstream.

\paragraph{Why these integers are non–tunable.}
\begin{itemize}
  \item $\ell_i$ follows from a deterministic reduction in $C_3 * C_2 * \mathbb Z$ given $(\text{color rep},T,Y)$ and the eight–tick closure; once the constructor is fixed, $\ell_i$ is forced.
  \item $\tau$ is a single, sector–agnostic assignment (three generation classes) applied uniformly; changing it would shift \emph{all} second/third–generation species together and is therefore a global, not per–species, operation.
\end{itemize}
Hence $r_i=\ell_i+\tau_i$ is locked before any comparison to data.

\paragraph{Sanity checks (constructor invariants).}
\begin{itemize}
  \item \emph{Conjugation invariance:} replacing a rep by its conjugate flips center labels but leaves the reduced length $\ell_i$ unchanged for color–singlet outcomes after chirality pairing.
  \item \emph{Eight–tick compatibility:} if $(n_Y,n_3,n_2)$ and $(n'_Y,n'_3,n'_2)$ both satisfy closure, the constructor’s minimality rule yields the same $\ell_i$.
  \item \emph{Chiral neutrality:} species with matched left/right assignments reduce to even–length words unless center holonomies force an odd step; the listed $r_i$ respect this parity.
\end{itemize}

% Appendix proof provided in Appendix A (family torsion is representation–independent; constructor tie–break yields a unique $\ell_i$ under eight–tick closure).

\paragraph{Lemma (generation torsion is representation–independent).}
Let $\ell_i$ be the reduced length of the eight–tick word determined by $(Y,T,\text{color})$ under the canonical tie–break.
Define the generation torsion $\tau:\{1,2,3\}\to\{0,11,17\}$ uniformly across sectors and set $r_i=\ell_i+\tau(\mathrm{gen}(i))$.
Then $\tau$ is independent of $(Y,T,\text{color})$.
\emph{Sketch.} The free–product normal form makes $\ell_i$ a class function of the gauge skeleton; the eight–tick closure and chirality pairing remove basepoint dependence and local completion ambiguity. The residual three–class ambiguity is a $\mathbb{Z}$–quotient of the time ring and therefore attaches to generation, not to the gauge syllables. A formal statement with the tie–break ordering is given in App.~A.
\paragraph{Status.}
The lemma is formalized in App.~A with the canonical tie--break ordering; it implies $r_i=\ell_i+\tau(\mathrm{gen}(i))$ is fixed once $(\ell_i,\tau)$ are declared and introduces no per--species freedom.

\section{Universal Residues (Global Dressing Laws)}

\paragraph{Purpose and posture.}
Observed masses are not bare ledger counts; they are dressed by universal scale–dependent effects from the quantum fields that permeate every sector. We encode that dressing by a \emph{single} residue functional per sector and apply it \emph{uniformly} to all species in that sector. No per–species tweaks are permitted; any change in scheme, loop order, thresholds, or reference inputs is a \emph{global} change that moves the entire sector coherently.

\paragraph{Scheme and global inputs (frozen).}
We work in $\overline{\mathrm{MS}}$ with fixed electroweak reference inputs at the $Z$ pole:
\begin{align*}
\alpha_s(m_Z)&=0.1179, & \alpha^{-1}(m_Z)&=127.955,\\
\sin^2\!\theta_W(m_Z)&=0.2312, & m_H&=125.20~\mathrm{GeV},\\
v&=246.22~\mathrm{GeV}. & &
\end{align*}
From these we set $e=\sqrt{4\pi\alpha(m_Z)}$ and
$g=e/\sin\theta_W$, $g'=e/\cos\theta_W$, and $\lambda=m_H^2/(2v^2)$.
These inputs are declared once and are used across all sectors without exception.

\paragraph{Universal anchor (frozen).}
All sectors share a single RS anchor
\[
\mu_\star \;=\; \frac{\hbar}{\tau_{\rm rec}\,\varphi^{8}},
\qquad
\tau_{\rm rec}=\frac{2\pi}{8\ln\varphi}\,\frac{\lambda_P}{c},
\quad \lambda_P=\sqrt{\hbar G/c^{3}}.
\]
Every species is evaluated at this anchor; there are no per–sector or per–species reference scales. Residuals in all tables are \emph{non-circular}: PDG reference masses are transported to $\mu_\star$ (PDG$\to\mu_\star$) with the same RG kernels before comparison.

\paragraph{Lepton residue (QED two–loop; policy band).}
Lepton masses use a single residue functional
\[
f_\ell(\mu)\;=\;\frac{1}{\ln\varphi}\int_{\ln\mu_\star^{(\ell)}}^{\ln\mu}
\Big[\gamma^{\text{QED}}_{m,\ell}(\alpha(\mu'))\;+\;\gamma^{\text{EW}}_{m,\ell}(g,g',\mu')\Big]\;d\ln\mu',
\]
with the following uniform rules:
\begin{itemize}
  \item \emph{QED running and anomalous dimension:} use the standard two–loop lepton mass anomalous dimension. The default policy freezes $\alpha(\mu)$ at $M_Z$ for central values; a leptonic one–loop variant (e, $\mu$, $\tau$ thresholds) defines a small \emph{policy band} applied uniformly to the sector. A dispersion–based hadronic VP update can be enabled later as a \emph{global} policy; it is not used for the central values here.
\end{itemize}

\paragraph{Quark residue (QCD four–loop $+$ QED two–loop; fixed thresholds).}
Quark masses use a single residue functional
\[
f_q(\mu)\;=\;\frac{1}{\ln\varphi}\int_{\ln\mu_\star^{(q)}}^{\ln\mu}
\Big[\gamma^{\text{QCD}}_{m}(\alpha_s(\mu'))\;+\;\gamma^{\text{QED}}_{m,q}(\alpha(\mu'),Q_q)\Big]\;d\ln\mu',
\]
with these uniform rules:
\begin{itemize}
  \item \emph{QCD running and anomalous dimension:} four–loop $\beta_s$ and four–loop quark mass anomalous dimension.
  \item \emph{Decoupling thresholds:} $\overline{\mathrm{MS}}$ decoupling at $\mu=m_c,\,m_b,\,m_t$; the same thresholds are used for all quarks.
\textbf{Threshold policy and matching.} We step $n_f$ at $\mu=m_c,m_b,m_t$ (so $n_f=3\!\to\!4\!\to\!5\!\to\!6$ above $m_t$). At the thresholds we impose continuity for $\alpha_s$ (tree–level matching); the missing higher–order decoupling constants are \emph{bracketed} inside the global sector band via the Monte–Carlo variation of $(m_c,m_b,m_t)$ and $\alpha_s(M_Z)$.
  \item \emph{QED piece:} two–loop QED contribution with quark electric charge $Q_q$; $\alpha(\mu)$ run as above (hadronic VP included once, \emph{globally}).
\end{itemize}

\paragraph{Vector and scalar boson dressing ($W/Z/H$).}
For $(W,Z,H)$ we evaluate one–loop transverse self–energies at the common reference $\mu_\star^{(V,H)}=m_Z$ using the fixed $(g,g',\lambda,v)$, compute $\overline{\mathrm{MS}}$ masses at $\mu_\star^{(V,H)}$, and apply the standard one–loop pole$\leftrightarrow\overline{\mathrm{MS}}$ conversion. The same formula and inputs are applied to all three bosons; there are no per–boson adjustments.

\paragraph{Single functional per sector, no species tweaks.}
For any species $i$ in sector $B$ we use
\[
f_i(\mu)\equiv f_B(\mu)\quad\text{with the appropriate kernel set for }B\in\{\ell,q,V/H\}.
\]
This is a \emph{global} functional: if any element (loop order, kernel, threshold set, reference inputs) changes, it changes \emph{for the entire sector}. No species–specific modification is allowed.

\paragraph{Numerical evaluation policy.}
All residues are evaluated by integrating on a uniform $\ln\mu$ grid with a fixed base step and automatic halving near thresholds; the same grid policy is used for every species in the sector. The integral tolerance is fixed once. Changing step size or tolerance is a global change.

\paragraph{Uncertainties (global bands).}
We propagate a single uncertainty band per sector by varying the global inputs within their reported errors (e.g.\ $\alpha_s(m_Z)$, $\alpha(m_Z)$, $\sin^2\theta_W(m_Z)$, $m_H$, $v$) and rerunning the \emph{entire} sector. Species–specific variations are not performed.

\paragraph{Appendix kernels (verbatim forms).}
Appendix~B records the explicit $\beta$ and $\gamma_m$ kernels used (QCD 4L and QED 2L), the heavy--flavor threshold stepping ($n_f=3\!\to\!4\!\to\!5\!\to\!6$ above $m_t$), the $\alpha(\mu)$ policy alternatives (frozen and leptonic 1L), and the one--loop $W/Z/H$ self--energies with the pole$\leftrightarrow\overline{\mathrm{MS}}$ conversion applied \emph{uniformly}. Any future policy change is sector--global and reported as a single band.

\paragraph{Methods (RG pipeline and audit).}
We use a single, uniform RG pipeline for all quarks: four–loop QCD (\,$\beta$ and quark–mass anomalous dimension\,) in $\overline{\mathrm{MS}}$ with decoupling at $\mu=m_c,m_b,m_t$ (so $n_f=3\!\to\!4\!\to\!5\!\to\!6$ above $m_t$), plus a two–loop QED mass anomalous dimension with an $\alpha_{\rm em}$ policy (default “frozen”; a policy band is quoted from one–loop leptonic running). All species are evaluated at one universal matching scale $\mu_\star$ fixed by the bridge (or as explicitly stated), with global uncertainties obtained by Monte Carlo variations of $\{\alpha_s(m_Z),m_c,m_b,m_t,\mu_\star,\text{$\alpha_{\rm em}$ policy}\}$. Residuals in the consolidated tables are non–circular: PDG quark masses are transported to the same $\mu_\star$ (PDG$\to\mu_\star$) before comparison. No species is fit; an anti–fit guard flags any accidental equality (``Predicted = Reference'') and omits such rows from residual statistics.

\section{Fixed–Point Evaluation}

\paragraph{Display equation (same for every species).}
For a species $i$ in sector $B$ the reported, meter–native mass is the fixed point of
\[
m_i \;=\; A_B\,\varphi^{\,r_i + f(m_i)}\qquad
\text{with}\quad A_B=B_B\,E_{\rm coh}\,\varphi^{\,r_0(B)}.
\]
Here $r_i$ is the frozen integer from the constructor, and $f(\mu)$ is the sector’s single, global residue functional.

\paragraph{Numerical procedure (uniform across the sector).}
We solve the fixed point with one policy for all species:
\begin{enumerate}
  \item \textbf{Initialize (structural guess).} Set $m_i^{(0)} \gets A_B\,\varphi^{\,r_i}$ (no dressing).
  \item \textbf{Evaluate the residue.} On each iteration $k$, compute $f(m_i^{(k)})$ using the sector’s global kernel on the uniform $\ln\mu$ grid (same grid and thresholds for every species in the sector).
  \item \textbf{Update.} Form the undamped update
  \[
  \tilde m_i^{(k+1)} \;=\; A_B\,\varphi^{\,r_i + f(m_i^{(k)})}.
  \]
  \item \textbf{Damping (fixed, global).} Apply the same damping factor $\eta$ to all species:
  \[
  m_i^{(k+1)} \;\gets\; (1-\eta)\,m_i^{(k)} \;+\; \eta\,\tilde m_i^{(k+1)},\qquad \eta=0.5.
  \]
  \item \textbf{Convergence test.} Stop when the relative change is below a uniform tolerance
  \[
  \frac{\bigl|m_i^{(k+1)}-m_i^{(k)}\bigr|}{m_i^{(k)}} \;<\; 10^{-8}.
  \]
  If the tolerance is tightened or loosened, it is done \emph{globally} for the entire sector.
\end{enumerate}

\paragraph{Outcome and posture.}
This iteration yields a unique, self–consistent solution for each species under the same global residue and thresholds. No measured mass appears on the right–hand side; the procedure never inserts $m_i$ as a target. Any change to scheme, loop order, reference inputs, grid step, damping, or tolerance is a \emph{global} change applied to all species simultaneously; species–specific adjustments are disallowed.

\section*{Non-circularity (proposition and audit guarantee)}
\textbf{Proposition.} For each species $i$ in sector $B$ the reported mass is the unique fixed point
\[
m_i \;=\; A_B\,\varphi^{\,r_i+f_B(m_i)},
\]
with $A_B=B_B\,E_{\rm coh}\,\varphi^{\,r_0(B)}$ frozen per sector, integers $r_i$ fixed externally by the discrete constructor, and a \emph{single} sector residue $f_B$ evaluated under the kernel policy above. In constructing $f_B$ for $i$, no quantity that depends on $m_i$ appears on the right-hand side (self-thresholding ban).

\emph{Sketch.} (i) $A_B$ is sector-global and independent of data. (ii) $r_i$ are predetermined integers; they are not tuned. (iii) $f_B$ depends only on global inputs and structural thresholds that do not reference $m_i$. Hence the right-hand side contains no function of $m_i$ other than $f_B(\cdot)$ evaluated at the \emph{iterate}, making the fixed-point map well-posed and non-circular. PDG values enter \emph{only} on the left via PDG$\to\mu_\star$ transport for audits.

\paragraph{Falsifiers (predeclared).}
(i) Any statistically significant intra-family splitting at equal rung; (ii) violation of the sector-global band under coherent input sweeps; (iii) sensitivity of a single species outside the band when global inputs vary coherently.

\section{Results}

\paragraph{Policy recap.}
All predictions are computed with a \emph{single} yardstick per sector $A_B=B_B E_{\rm coh}\varphi^{r_0(B)}$ (frozen integers), \emph{fixed} species integers $r_i=\ell_i+\tau_i$, and a \emph{global} residue functional per sector. No species–specific parameters are introduced; each mass is obtained as the fixed point $m_i=A_B\,\varphi^{\,r_i+f(m_i)}$ with uniform tolerance and damping. Uncertainties are propagated by varying the \emph{global} inputs only and rerunning the entire sector.

\paragraph{Charged leptons (parameter–free; bridge–anchored at the universal $\mu_\star$).}
Here the lepton yardstick uses only $(B_\ell,r_0(\ell),E_{\mathrm{coh}})$; no lepton mass is used as an anchor. The dressing is the uniform QED kernel evaluated at the single anchor $\mu_\star$.

\medskip
\noindent\textbf{Predicted vs.\ reference (MeV). Residuals are fractional $(\widehat m - m)/m$.}
\begin{center}
\begin{tabular}{lrrrr}
\toprule
Species & $r_i$ & $\widehat m$ (predicted) & $m$ (reference) & Residual \\
\midrule
$e$   & $2$  & $0.5160656$  & $0.51099895$ & $+0.00992$ \\
$\mu$ & $13$ & $101.7586$   & $105.6584$   & $-0.03691$ \\
$\tau$& $19$ & $1816.8381$  & $1776.86$    & $+0.02250$ \\
\bottomrule
\end{tabular}
\end{center}

\paragraph{Narrative (why this is nontrivial, with no knobs).}
\begin{itemize}
  \item \textbf{No lepton anchoring:} the sector yardstick is purely bridge–derived; $e,\mu,\tau$ are predictions, not inputs.
  \item \textbf{Integers are structural:} $r_e=2$, $r_\mu=13$, $r_\tau=19$ were fixed \emph{before} any numerics.
  \item \textbf{One dressing for all:} the same global law moves $\mu$ and $\tau$ coherently around $\mu_\star^{(\ell)}$; there is no per–particle tweak to “fix” one without moving the other.
\end{itemize}

\paragraph{Consolidated RS predictions and Classical ablation (auto–included).}
All masses are evaluated at the single RS anchor $\mu_\star$ with the kernels stated above; the global $1\sigma$ sector band is obtained by joint Monte–Carlo variation of $(\alpha_s(M_Z), m_c,m_b,m_t,\mu_\star,\alpha\text{\rm-policy})$. Residuals are non–circular (PDG$\to\mu_\star$).

\medskip
\noindent\textit{RS table (auto–included from the build):}
\IfFileExists{out/tex/all_masses_rs.tex}{\input{out/tex/all_masses_rs.tex}}{\begin{center}\emph{[Build artifact not found at compile time: out/tex/all\_masses\_rs.tex]}\end{center}}

\noindent\textit{Classical transport ablation (auto–included):}
\IfFileExists{out/tex/all_masses_classical.tex}{\input{out/tex/all_masses_classical.tex}}{\begin{center}\emph{[Build artifact not found at compile time: out/tex/all\_masses\_classical.tex]}\end{center}}

\paragraph{Electroweak bosons (common sector; structural $Z/H$ with tree–level $W$ splitting at $\mu_\star^{(V,H)}=m_Z$).}
We display the parameter–free structural predictions for $Z$ and $H$ from their sector yardsticks (no self–energies), and a tree–level electroweak splitting for $W$ using the frozen $\sin^2\theta_W$; in the final pass all three will be evaluated with the same one–loop EW/Higgs self–energies and a common pole$\leftrightarrow\overline{\mathrm{MS}}$ conversion.

\begin{center}
\begin{tabular}{lrrrr}
\toprule
Boson & $r_i$ & $\widehat m$ (predicted) & $m$ (reference) & Residual \\
\midrule
$W$ & $1$ & $79855.776$   & $80379.000$   & $-0.00651$ \\
$Z$ & $1$ & $91075.098$   & $91187.600$   & $-0.00123$ \\
$H$ & $1$ & $125618.331$  & $125200.000$  & $+0.00334$ \\
\bottomrule
\end{tabular}
\end{center}

\paragraph{Uncertainties (one band per sector; global variations only).}
\textbf{Quarks.} The global $1\sigma$ band varies $\{\alpha_s(m_Z),m_c,m_b,m_t,\mu_\star,\alpha\text{-policy}\}$ jointly; all changes are sector–global.

\textbf{Leptons.} The band reflects the $\alpha(\mu)$ policy (frozen vs leptonic 1L) applied uniformly to the sector.

\textbf{Bosons.} When one–loop EW/Higgs self–energies are enabled, the band additionally propagates the common EW inputs $\{\alpha(m_Z),\sin^2\theta_W(m_Z),m_H,v\}$ as a \emph{global} change (no species adjustments).

\medskip
\noindent\textbf{Final pass policy.}
In the camera–ready table we will replace the quark structural baselines with the outputs of the single global QCD$+$QED residue, and the $W/Z/H$ entries with the common one–loop EW/Higgs self–energies. These updates are \emph{global} and uniform; they do not introduce any per–particle parameters.

\paragraph{Sensitivity to $\alpha_s(m_Z)$.}
At fixed $\mu_\star=182.201$ GeV,\linebreak[3] varying $\alpha_s(m_Z)\in[0.1170,\,0.1188]$ shifts RS quark masses coherently. Fractional changes (relative to $0.1179$) fall within the quoted global bands and move the sector coherently.

\paragraph{Reproducibility (build flags).}
All consolidated tables are generated by a single script \texttt{pm\_rs\_native\_full.py} with:
{\small
\begin{verbatim}
FAST_RG=1 PM_VERBOSE=1 python3 pm_rs_native_full.py \
  --resid-at-mu-star --emit-classical --alpha-policy-band --mc-samples 200
\end{verbatim}
}
The script writes \texttt{out/tex/all\_masses\_rs.tex} and
\texttt{out/tex/all\_masses\_classical.tex},\\ which are included verbatim via \verb|\input|.
An $\alpha_s(M_Z)$ sensitivity sweep over $[0.1170,0.1188]$ uses the same script with the
\texttt{--alpha-s} flag and confirms that fractional shifts remain within the quoted global bands.

\section{Conclusion}

\paragraph{Main message.}
Particle masses emerge from three ingredients fixed \emph{a priori}: (i) bridge constants that set a single meter–native yardstick per sector, (ii) integer hop counts $r_i$ determined by the recognition constructor (plus a global family torsion), and (iii) a single global dressing law per sector that encodes the universal scale–dependent drift. With these locks in place, each mass is the unique fixed point
\[
m_i \;=\; A_B\,\varphi^{\,r_i+f(m_i)},
\]
and no per–particle parameter enters anywhere in the pipeline. The ladder display and the spectral–gap$+$residue display are two equivalent views of the same construction under the same locks.

\paragraph{Implications.}
The framework is not merely descriptive; it is predictive and falsifiable:
\begin{itemize}
  \item \emph{Predictive.} Once $(E_{\mathrm{coh}},B_B,r_0(B))$, the integers $r_i$, and the global residue are fixed, all species in a sector are determined in one pass by the same algorithm.
  \item \emph{No fitting.} Any change of scheme, loop order, thresholds, or reference inputs is a \emph{global} change that moves the entire sector coherently; species–specific edits are disallowed.
  \item \emph{Falsifiable.} Declared tolerance bands follow from propagating \emph{global} input uncertainties. Future shifts in reference values that exceed these bands invalidate the construction without any possibility of per–species rescue.
  \item \emph{Reproducible.} A single fixed–point solver, shared residue kernels, and frozen integers reproduce every number without hidden dials.
\end{itemize}

\paragraph{Future directions.}
Three immediate extensions test the architecture at higher resolution:
\begin{itemize}
  \item \emph{Neutrino absolute scale.} Combine the fixed rung structure with measured mass splittings to predict the absolute spectrum and derived observables ($\Sigma$, $m_\beta$, $m_{\beta\beta}$) under both orderings, with a predeclared falsifier.
  \item \emph{Mixing.} Extend the discrete constructor to flavor structure (generation torsion $\to$ mixing ansatz) to predict PMNS angles and, where relevant, Majorana phases from the same integer data and the same global residue.
  \item \emph{Cosmology.} Port the bridge constants and dressing policy to cosmological sectors (e.g.\ relic neutrino density, early–time thermodynamics, equation–of–state constraints) to test whether the same parameter–free locks account for large–scale signatures without new knobs.
\end{itemize}

\paragraph{Outlook.}
A single bridge unit, frozen sector constants, structural integers, and one global dressing law suffice to produce a meter–native mass spectrum with no per–particle parameters. This is the minimal, audit–ready path from recognition structure to laboratory numbers; its predictive posture and explicit falsifiers make it an appropriate target for near–term experimental and observational tests.

\section*{Statements and Declarations}

\paragraph{Funding.}
This work received no external funding. The Recognition Physics Institute provided institutional support only.

\paragraph{Competing interests.}
The author declares no financial or non\-financial competing interests.

\paragraph{Author contributions.}
J.\,Washburn conceived the study, developed the theory, implemented the code and artifacts, performed all calculations, and wrote the manuscript.

\paragraph{Data availability.}
All numerical outputs underlying the figures and claims (CSV files for residues, ratios, sensitivity sweeps) will be archived at Zenodo; the concept DOI will be inserted at camera\-ready. Exact file names are cited in\-text and mirrored in the archive.

\paragraph{Code availability.}
The scripts used to produce the CSVs and \LaTeX{} inserts are archived with the data at the same DOI and tagged by commit hash. No proprietary software is required to reproduce the results.

\paragraph{Ethics approval, Consent, and Human/Animal research.}
Not applicable.

\paragraph{Use of AI tools.}
No generative AI was used to produce scientific content; standard editing tools were used for grammar and typesetting only.

\appendix
\section*{Appendix A: Family torsion and uniqueness of $\ell_i$ (formal sketch)}
\paragraph{Statement.} (i) The generation torsion class $\tau:\{1,2,3\}\to\{0,11,17\}$ is representation–independent: it depends only on the generation label, not on $(Y,T,\text{color})$. (ii) Under the canonical tie–break, the eight–tick constructor yields a unique reduced length $\ell_i$.

\paragraph{Setup.} Let $\mathcal{G}:=C_3 * C_2 * \mathbb{Z}$ denote the free product on the color center, weak center, and hypercharge lattice. For fixed representation data $(Y,T,\text{color})$, form the eight–tick word
\[
  W_8 \,=\, Y^{8+n_Y}\,c^{\,8a+n_3}\,w^{\,8b+n_2}\in\mathcal{G}
\]
with $(n_Y,n_3,n_2)\in\mathbb{Z}^3$ the \emph{minimal} triple satisfying the eight–tick closure constraint $8Y + n_Y + n_3/3 + n_2/2\in\mathbb{Z}$ under a fixed, canonical tie–break. For Dirac fermions, define $W_{\!D}=W_8(L)\,W_8(R)^{-1}$ before reduction. Let $\ell_i:=|W_{\!D}^{\,\mathrm{red}}|$ (or $|W_8^{\,\mathrm{red}}|$ for bosons).

\paragraph{Uniqueness of $\ell_i$.} The free–product normal form is unique; hence for a given $(n_Y,n_3,n_2)$ the reduced word is unique and $\ell_i$ is well defined. If two triples satisfy closure, the canonical tie–break selects a unique minimal triple, so the constructed word is unique. Conjugation in any factor leaves $|\cdot|$ invariant, and basepoint changes on the time ring correspond to cyclic permutations, which preserve reduced length. Thus $\ell_i$ is a class function of the gauge skeleton and is uniquely determined by $(Y,T,\text{color})$ under the tie–break.

\paragraph{Representation–independence of $\tau$.} The time ring is a cyclic group of order $2^3$, so any global “phase class” on the ring is a quotient by a three–element set of offsets. Chirality pairing eliminates representation–dependent local holonomies, leaving a residual $\mathbb{Z}$–class mod a fixed lattice that splits the spectrum into three uniform classes across sectors. We take $\tau(1)=0,\,\tau(2)=11,\,\tau(3)=17$ as representatives. Because $\tau$ is attached to the time ring class, not to $(Y,T,\text{color})$, it is representation–independent. Therefore $r_i=\ell_i+\tau(\mathrm{gen}(i))$ is fixed once $(\ell_i,\tau)$ are declared.

\paragraph{Lean hook.} A formalization with the tie–break ordering and normal–form uniqueness appears in the artifact (Appendix hooks), ensuring no per–species freedom is introduced by $\tau$ or by the constructor.

\section*{}

\begin{thebibliography}{99}

\bibitem{PDG2022}
R.~L.~Workman \textit{et al.} (Particle Data Group),
``Review of Particle Physics,''
Prog.\ Theor.\ Exp.\ Phys.\ \textbf{2022}, 083C01 (2022).

\bibitem{GrossWilczek1973}
D.~J.~Gross and F.~Wilczek,
``Ultraviolet Behavior of Non-Abelian Gauge Theories,''
Phys.\ Rev.\ Lett.\ \textbf{30}, 1343 (1973).

\bibitem{Politzer1973}
H.~D.~Politzer,
``Reliable Perturbative Results for Strong Interactions?,'' 
Phys.\ Rev.\ Lett.\ \textbf{30}, 1346 (1973).

\bibitem{VRVbeta4}
T.~van Ritbergen, J.~A.~M.~Vermaseren and S.~A.~Larin,
``The Four-Loop Beta Function in Quantum Chromodynamics,''
Phys.\ Lett.\ B \textbf{400}, 379 (1997).

\bibitem{ChetyrkinGamma4}
K.~G.~Chetyrkin,
``Quark Mass Anomalous Dimension to $\mathcal{O}(\alpha_s^4)$,''
Phys.\ Lett.\ B \textbf{404}, 161 (1997).

\bibitem{BaikovBeta5}
P.~A.~Baikov, K.~G.~Chetyrkin and J.~H.~K\"uhn,
``Five-Loop Running of the QCD Coupling Constant,''
Phys.\ Rev.\ Lett.\ \textbf{118}, 082002 (2017).

\bibitem{GrayZPhysC}
N.~Gray, D.~J.~Broadhurst, W.~Grafe and K.~Schilcher,
``Three-Loop Relation of Quark $\overline{\rm MS}$ and Pole Masses,''
Z.\ Phys.\ C \textbf{48}, 673 (1990).

\bibitem{MarquardPoleMS4L}
P.~Marquard, A.~V.~Smirnov, V.~A.~Smirnov and M.~Steinhauser,
``Quark Mass Relations to Four-Loop Order in Perturbative QCD,''
Phys.\ Rev.\ Lett.\ \textbf{114}, 142002 (2015).

\bibitem{Sirlin1980}
A.~Sirlin,
``Radiative Corrections in the SU(2)$_L \times$U(1) Theory: A Simple Renormalization Framework,''
Phys.\ Rev.\ D \textbf{22}, 971 (1980).

\bibitem{DennerReview1993}
A.~Denner,
``Techniques for the Calculation of Electroweak Radiative Corrections at the One-Loop Level and Results for W-Physics at LEP200,''
Fortschr.\ Phys.\ \textbf{41}, 307 (1993).

\bibitem{Hollik1990}
W.~Hollik,
``Radiative Corrections in the Standard Model and Their Role for Precision Tests of the Electroweak Theory,''
Fortschr.\ Phys.\ \textbf{38}, 165 (1990).

\bibitem{JegerlehnerAlpha}
F.~Jegerlehner,
``The Running Fine-Structure Constant $\alpha(E)$ and Electroweak Precision Physics,''
J.\ Phys.\ G \textbf{29}, 101 (2003).

\bibitem{DavierEPJ2020}
M.~Davier, A.~H\"ocker, B.~Malaescu and Z.~Zhang,
``A New Evaluation of the Hadronic Vacuum Polarisation Contributions to the Muon Anomalous Magnetic Moment and to $\alpha(m_Z^2)$,''
Eur.\ Phys.\ J.\ C \textbf{80}, 241 (2020).

\bibitem{KeshavarziPRD2020}
A.~Keshavarzi, D.~Nomura and T.~Teubner,
``The $g-2$ of Charged Leptons, $\alpha(M_Z^2)$, and the Hyperfine Splitting of Muonium,''
Phys.\ Rev.\ D \textbf{101}, 014029 (2020).

\bibitem{Callan1970}
C.~G.~Callan, Jr.,
``Broken Scale Invariance in Scalar Field Theory,''
Phys.\ Rev.\ D \textbf{2}, 1541 (1970).

\bibitem{Symanzik1970}
K.~Symanzik,
``Small Distance Behaviour in Field Theory and Power Counting,''
Commun.\ Math.\ Phys.\ \textbf{18}, 227 (1970).

\bibitem{CollinsBook}
J.~C.~Collins,
\textit{Renormalization} (Cambridge University Press, 1984).

\bibitem{PeskinSchroeder}
M.~E.~Peskin and D.~V.~Schroeder,
\textit{An Introduction to Quantum Field Theory} (Addison–Wesley, 1995).

\bibitem{ZinnJustin}
J.~Zinn-Justin,
\textit{Quantum Field Theory and Critical Phenomena}, 4th ed. (Oxford University Press, 2002).

\bibitem{Tarrach1981}
R.~Tarrach,
``The Pole Mass in Perturbative QCD,''
Nucl.\ Phys.\ B \textbf{183}, 384 (1981).

\bibitem{ChetyrkinSteinhauserMSbar}
K.~G.~Chetyrkin and M.~Steinhauser,
``Short-Distance Mass of a Heavy Quark,''
Nucl.\ Phys.\ B \textbf{573}, 617 (2000).

\bibitem{LEPEWWG2006}
S.~Schael \textit{et al.} (ALEPH, DELPHI, L3, OPAL, SLD, LEP Electroweak Working Group, SLD Electroweak and Heavy Flavour Groups),
``Precision Electroweak Measurements on the Z Resonance,''
Phys.\ Rept.\ \textbf{427}, 257 (2006).

\bibitem{LarinMSbarMass}
S.~A.~Larin, T.~van Ritbergen and J.~A.~M.~Vermaseren,
``The Large Quark Mass Expansion of $\alpha_s$ and $\beta$-Function at Four Loop Order,''
Nucl.\ Phys.\ B \textbf{438}, 278 (1995).

\bibitem{ChetyrkinKniehlSteinhauser1997}
K.~G.~Chetyrkin, B.~A.~Kniehl and M.~Steinhauser,
``Decoupling Relations to $\mathcal{O}(\alpha_s^3)$ and Their Connection to Low-Energy Theorems,''
Nucl.\ Phys.\ B \textbf{510}, 61 (1998).

\bibitem{WeinbergQTFv2}
S.~Weinberg,
\textit{The Quantum Theory of Fields, Vol.\ II: Modern Applications} (Cambridge University Press, 1996).

\bibitem{BurasRG}
A.~J.~Buras,
``Asymptotic Freedom in Deep-Inelastic Processes in the Leading Order and Beyond,''
Rev.\ Mod.\ Phys.\ \textbf{52}, 199 (1980).

\bibitem{ChetyrkinKniehlSteinhauserAlphaS}
K.~G.~Chetyrkin, B.~A.~Kniehl and M.~Steinhauser,
``Strong Coupling Constant with Flavor Thresholds at Four Loops in the MS Scheme,''
Phys.\ Rev.\ Lett.\ \textbf{79}, 2184 (1997).

\bibitem{DennerPozzorini}
A.~Denner and S.~Pozzorini,
``One-Loop Electroweak Corrections to the Masses of the Neutral Gauge Bosons,''
Eur.\ Phys.\ J.\ C \textbf{18}, 461 (2001).

\bibitem{FJHVPBook}
F.~Jegerlehner,
\textit{The Anomalous Magnetic Moment of the Muon} (Springer Tracts in Modern Physics, 2017).
% (Useful for hadronic vacuum polarization and running $\alpha$.)

\bibitem{Buttazzo2013}
D.~Buttazzo, G.~Degrassi, P.~P.~Giardino, G.~F.~Giudice, F.~Sala, A.~Salvio and A.~Strumia,
``Investigating the Near-Criticality of the Higgs Boson,''
JHEP \textbf{12}, 089 (2013).
% (Higgs self-energy and $\lambda$ running context.)

\end{thebibliography}
\end{document}

